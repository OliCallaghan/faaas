\chapter{Design}

\section{Developer Considerations}
In alignment with the ethos of \faas{} whereby developer time is spent focusing on the business logic of their application rather than the infrastructure, the design of \faaasc{} should be such that it is easy to use and requires minimal developer effort to leverage the benefits of function splitting.

Since the primary code target of \faaasc{} is JavaScript (and by extension TypeScript) serverless functions, specifically ES6 syntax, \faaasc{} should be designed such that it integrates with existing tooling to deploy to \awslambda{}, specifically Serverless\cite{serverless-cli}.

\section{High-level Design}

\begin{figure}[htp]
    \centering
    \subfigure[\faas{} architecture from request to response]{
        \centering
        \begin{tikzpicture}[scale = 0.75, every node/.style={scale=0.75}]
            \input{node_modules/@faaas/arch-overview-source/assets/arch-overview-source.pgf}
        \end{tikzpicture}
    }\quad
    \subfigure[Split \faas{} handler using message passing to between split sections of function handler body]{
        \centering
        \begin{tikzpicture}[scale = 0.75, every node/.style={scale=0.75}]
            \input{node_modules/@faaas/arch-overview-split/assets/arch-overview-split.pgf}
        \end{tikzpicture}
    }
    \caption{\faas{} architecture enabling split function handlers}
\end{figure}

\begin{listing}[H]
  \inputminted{javascript}{node_modules/@faaas-bench/hello-seq/src/onHttpGetHello.trigger.ts}
  \caption{Typical serverless function handler interacting with a database via an ORM.}
\end{listing}

AST of the handler body

\begin{figure}[t]
    \includegraphics[width=\linewidth]{node_modules/@faaas-bench/hello-seq/assets/module.pdf}
    \caption{AST of handler body.}
    \label{fig:suites-hello-seq-module-ast}
\end{figure}
