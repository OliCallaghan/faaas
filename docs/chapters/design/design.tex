\chapter{Code splitting using dynamic response time modelling}

As detailed in Section \ref{sec:double-billing-problem}, serverless functions that interface with external asynchronous services are particularly vulnerable to the double billing problem.

To reduce billed idle time, we propose using code generation to split existing serverless functions around calls to asynchronous services, suspending billing for this period, and resume execution once the asynchronous result is ready.

Since \faaas{} billing is deterministic as described in Section \ref{sec:faas-billing-models}, we can accurately determine the cost of ending an invocation and reinvoking the continuation at a later time. This allows us to build a generic parameterised model described in Section \ref{sec:faas-param-cost-model} that can be applied to any \faas{} platform, given the set of parameters that accurately describe the platform's billing model.

Additionally, in Section \ref{sec:faas-async-service-response-time-modelling} we will introduce a model to estimate response times when interacting with asynchronous services, which in conjunction with the parameterised cost model, will be used to determine the profitability of code splitting in Section \ref{sec:faas-code-splitting-profitability}.

\section{Parameterised cost model for \faas{} platforms}
\label{sec:faas-param-cost-model}
\todo[inline]{Parameterised cost model}

\section{Asynchronous service response time modelling}
\label{sec:faas-async-service-response-time-modelling}
In this section we will develop a generic model that is able to estimate the response time of an asynchronous service.

\subsection{Persistent storage response time modelling}
\todo[inline]{Introduce how response times are modelled when interacting with persistent storage.}

\section{Code splitting cost estimation}
\todo[inline]{Probably should move the background chapter explaining the model to this section}

\section{Code splitting profitability analysis}
\label{sec:faas-code-splitting-profitability}
\todo[inline]{Introduce the cost/benefit model for deciding whether to code split.}

\subsection{Monitoring and strategy switching}
\todo[inline]{Discuss how the profitability model can be used to switch between code splitting strategies.}

\section{High-level Design}
The core principal underpinning the design of \faaas{} is to decompose each \faas{} invocation into as a set of function continuations. Similarly to how event driven runtimes such as \js{} handle continuations of asynchronous code with an event loop as described in Section \ref{sec:js-event-loop}, \faaas{} takes advantage of function splitting (via code generation) and message passing to register continuations to be executed with a continuation context.

\begin{figure}[htp]
    \centering
    \subfigure[\faas{} architecture from request to response]{
        \centering
        \begin{tikzpicture}[scale = 0.75, every node/.style={scale=0.75}]
            \input{node_modules/@faaas/arch-overview-source/assets/arch-overview-source.pgf}
        \end{tikzpicture}
    }\quad
    \subfigure[Split \faas{} handler using message passing to between split sections of function handler body]{
        \centering
        \begin{tikzpicture}[scale = 0.75, every node/.style={scale=0.75}]
            \input{node_modules/@faaas/arch-overview-split/assets/arch-overview-split.pgf}
        \end{tikzpicture}
    }
    \caption{\faas{} architecture enabling split function handlers}
\end{figure}

\begin{listing}[H]
  \inputminted{javascript}{node_modules/@faaas-bench/hello-seq/src/onHttpGetHello.trigger.ts}
  \caption{Typical serverless function handler interacting with a database via an ORM.}
\end{listing}

\section{Splitting Directive}

To reduce developer effort adapting existing functions to support function splitting, a custom directive is introduced into the function handler body, \verb|'use async'|. The purpose of this directive is to indicate to the \faaasc{} compiler that the function handler body should be split into multiple functions at this point, each of which is executed in separate \awslambda{} functions.

This splitting directive resembles the \verb|'use strict'| directive in JavaScript, which indicates that the code should be executed in strict mode. The \verb|'use async'| directive is a pragma that is not part of the JavaScript language, but is understood by the \faaasc{} compiler. Therefore as a result, unless the \faaasc{} compiler is used, the directive will be ignored by the JavaScript runtime, and so the same code can be run on any other \faas{} platform without modification.

\begin{listing}[H]
\begin{minted}[obeytabs=true,tabsize=2]{javascript}
export async function handler(_) {
  "use async";
  const foo = await bar();
}
\end{minted}
\caption{Example usage of the directive.}
\label{listing:use-async-simple-example}
\end{listing}

Immediately following the \verb|'use async'| directive, it is expected that a variable declaration is made and initialised to the resolved value from a promise constructed by calling a function, as seen in Listing \ref{listing:use-async-simple-example}.

\section{Capturing Scope}

Since from the developer's perspective, the function handler body is a single function, the \faaasc{} compiler must capture the scope of the function handler body at the point of the \verb|'use async'| directive. This is achieved by parsing the function handler body into an Abstract Syntax Tree (AST), and capturing any uses of now free variables beyond the split point.

This scope is captured into a serialized context, and stored so that the continuation can be executed with the correct scope. The continuation is invoked when the async request is completed.

\begin{figure}[t]
    \includegraphics[width=\linewidth]{node_modules/@faaas-bench/hello-seq/assets/module.pdf}
    \caption{AST of handler body.}
    \label{fig:suites-hello-seq-module-ast}
\end{figure}

\section{Splitting cost optimisation}
\todo[inline]{Discuss what you're going to discuss.}

\subsection{Splitting cost estimation}
\todo[inline]{Describe how the cost of splitting is estimated.}

\subsection{Splitting cost optimisation algorithm}
\todo[inline]{Describe the different algorithms used to optimise the cost of splitting.}

\section{Message passing}
