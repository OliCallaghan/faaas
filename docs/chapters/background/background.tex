\chapter{Background}

\section{Origins of Serverless}
\subsection{What is serverless?}
\todo[inline]{Cite page 168 of wardley maps book}
Serverless computing emerged in the mid-to-late 2000s\cite{wardleyWardleyMaps2022,IntroducingGoogleApp,patilServerlessComputingEmergence2021} as a new paradigm for deploying applications, with the emergence of low-cost public cloud providers\cite{patilServerlessComputingEmergence2021,BenjaminBlackEC2}. It allows developers to deploy applications without managing the underlying infrastructure, leading to much more scalable and cost-effective solutions.

Typically, resources such as VMs\cite{hoeferTaxonomyCloudComputing2010} are rented in sub-second increments, and with storage and network charged by total usage. This results in a Pay-as-you-go (PAYG) model that can adapt to highly variable workloads\cite{sehgalCostBillingPractices2023,hilleyCloudComputingTaxonomy2009}.

Serverless applications are usually composed of cloud provider managed services, such as databases, storage, and compute, and are often event-driven\cite{EventarcOverview,EventListenerAmazon,robeceOverviewAzureEvent2024}. This means that application logic can be invoked by events, such as HTTP requests, database changes, or file uploads. This allows the application to scale independently\cite{goniwadaCloudNativeArchitecture2022}, as the cloud provider can provision resources as needed.

\subsection{What is \faas{}?}
\todo[inline]{Define \faas{}, and discuss the benefits and drawbacks of using \faas{} architectures.}

\section{Public \faas{} providers}
\subsection{VM based providers}
\todo[inline]{Discuss VM based providers, and the benefits and drawbacks of using them.}

\subsection{Container based providers}
\todo[inline]{Discuss container based providers, and the benefits and drawbacks of using them.}

\section{Novel \faas{} approaches}
\subsection{Wasm based providers}
\todo[inline]{Discuss Wasm based providers, and the benefits and drawbacks of using them.}

\subsection{V8 isolate based providers}
\todo[inline]{Discuss V8 isolate based providers, and the benefits and drawbacks of using them.}

\section{AWS Lambda}

AWS Lambda is by far the most popular\cite{eismannReviewServerlessUse2020,StateServerlessDatadog} \faas{} platform used by developers to deploy serverless applications.

\subsubsection{System architecture}

The AWS Lambda system architecture is centered around the concept of Firecracker MicroVMs\cite{agacheFirecrackerLightweightVirtualization2020}.

\begin{figure*}[t]
    \includegraphics[width=\linewidth]{node_modules/@faaas/aws-lambda-exec-env/assets/aws-lambda-exec-env.pdf}
    \caption{AWS Lambda Execution Environment}
    \label{fig:aws-lambda-exec-env}
\end{figure*}

\subsection{Concurrent executions}

Each invocation of a lambda function executes independently inside of it's own Firecracker MicroVM, in what is known as a slot.

Each MicroVM provides a slot which can handle a single invocation, however once this invocation completes, the slot can be used by another invocation.

\subsection{Pricing structure}

Billed for execution time from start to finish, since a MicroVM is provisioned the entire time.

\section{V8 and the Event Loop}
\label{sec:js-event-loop}

\todo[inline]{Outline the ins and outs of how event loops work. Also need to discuss re: function continuations.}

\section{Continuations and Continuation Passing Style}
A function continuation is a concept in programming\cite{sussmanSCHEMEInterpreterExtended1975}, reified as a datastructure encapsulating an execution state, and a function pointer that can be called to resume execution. They are used extensively across many languages, for example, they form the underpinnings of Rust's Futures API.
