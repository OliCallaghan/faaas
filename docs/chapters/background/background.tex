\chapter{Background}

\section{Terminology}
\subsection{What is serverless?}
\todo[inline]{Cite page 168 of wardley maps book}
Serverless computing emerged in the mid-to-late 2000s\cite{wardleyWardleyMaps2022,IntroducingGoogleApp,patilServerlessComputingEmergence2021} as a new paradigm for deploying applications, with the emergence of low-cost public cloud providers\cite{patilServerlessComputingEmergence2021,BenjaminBlackEC2}. It allows developers to deploy applications without managing the underlying infrastructure, leading to much more scalable and cost-effective solutions.

Typically, resources such as VMs\cite{hoeferTaxonomyCloudComputing2010} are rented in sub-second increments, and with storage and network charged by total usage. This results in a Pay-as-you-go (PAYG) model that can adapt to highly variable workloads\cite{sehgalCostBillingPractices2023,hilleyCloudComputingTaxonomy2009}.

Serverless applications are usually composed of cloud provider managed services, such as databases, storage, and compute, and are often event-driven\cite{EventarcOverview,EventListenerAmazon,robeceOverviewAzureEvent2024}. This means that application logic can be invoked by events, such as HTTP requests, database changes, or file uploads. This allows the application to scale independently\cite{goniwadaCloudNativeArchitecture2022}, as the cloud provider can provision resources as needed.

\subsection{What is a VM?}
\todo[inline]{Talk about how VMs act as an abstraction over physical hardware and can be provisioned to run applications. Typical startup times can vary based on how heavy-weight the VM is, typically ranging from seconds to minutes.}

\subsection{What is a container?}
\todo[inline]{Introduce containers as now the defacto standard of deployment of applications to cloud and serverless environments.}

\subsubsection{CGroups and Namespaces}
\todo[inline]{Talk about: CGroups, Namespaces and how they provide isolation mechanisms for containers.}

\subsubsection{Container Runtimes}
\todo[inline]{Talk about: runC, rkt, containerd, lxc.}

\subsection{What is \faas{}?}
Derived from the success of serverless computing, \faas{} rose to prominence in the mid-2010s\cite{AmazonWebServices2014,azureAnnouncingGeneralAvailability2016}. It allows developers to write code as functions that are executed in sandboxed environments in response to events, such as HTTP requests, database changes, or file uploads\cite{EventarcOverview,EventListenerAmazon,robeceOverviewAzureEvent2024}. Typically, serverless functions are billed for the duration of their execution in addition to a flat invocation cost, allowing for fine-grained billing of resources\cite{bortoliniInvestigatingPerformanceCost2020}. Additionally, since the underlying infrastructure of the function is abstracted away from the developer, the responsibility for scaling the function is moved to the cloud provider.

\section{\faas{} isolation mechanisms}
Typically, multi-tenany public \faas{} providers can be characterised into two categories: providers offering VM based isolation\cite{agacheFirecrackerLightweightVirtualization2020} and providers offering container based isolation\cite{GVisor}. Both types provide differing levels of security guarantees, and have different performance characteristics.

\subsection{VM based isolation}
VM based isolation is typically considered a much more secure form of isolation between \faas{} functions on multitenant public clouds\cite{jithinVirtualMachineIsolation2014}, however it typically has the highest overhead, since each function runs in its own virtual machine, with its own kernel. This overhead usually manifests in higher cold start times, however there has been considerable work carried out to reduce this overhead\cite{razaviPrebakedUVMsScalable2015,agacheFirecrackerLightweightVirtualization2020,dawXanaduMitigatingCascading2020,oliverstenbomRefunctionEliminatingServerless2019}.

Kata Containers\cite{KataContainersOpen} provides a container-like interface to VMs, interfacing with existing hypervisors such as Firecracker\cite{agacheFirecrackerLightweightVirtualization2020}, QEMU\cite{QEMU} and Cloud Hypervisor\cite{CloudhypervisorCloudhypervisorVirtual} (previously NEMU\cite{IntelNemu2024}).

Firecracker \cite{agacheFirecrackerLightweightVirtualization2020} is a lightweight KVM\cite{KVM} based Virtual Machine Monitor (VMM) allowing MicroVM provisioning in the order of \qtyrange{125}{150}{\ms} that provides a secure and fast environment for running \faas{} functions.

MicroVMs are intended to be lightweight, and provide a secure environment for running functions, with a fast boot time. Firecracker is written in Rust and unlike QEMU, provides secure, fast and stripped back devices to the guest VM in order to both reduce the attack surface and improve performance\cite{jainStudyFirecrackerMicroVM2020}.

LightVM redesigns the Xen Hypervisor's control plane such that VMs can boot in the order of \qtyrange{2}{3}{\ms}\cite{mancoMyVMLighter2017}, and achieveing a much higher VM density on a host.

\subsection{Container based isolation}
In contrast to VM based isolation, container based isolation is typically considered to be less secure\cite{DemystifyingContainerVs}, however it has a much lower overhead, since each function runs in its own container sharing a kernel with the host\cite{WhatContainerDocker}. From a security perspective, any kernel vulnerabilities could potentially be exploited by a malicious container to escape the sandbox and gain access to the host\cite{linMeasurementStudyLinux2018}.

GVisor is a container runtime that aims to improve security for running containers in a multitenant environment\cite{GVisor}. It implements a userspace `application kernel' that intercepts system calls made by untrusted container, providing a layer of abstraction between any possible vulnerabilities in the host kernel and the code executing in the untrusted container.

Whilst cold-start times of container-based \faas{} isolation are considerably lower than VM based isolation methods, cold-starts times can vary based on a variety of factors. SOCK reduces cold-start times by caching common dependencies in container images, reducing the image footprint and improving cold start times\cite{oakesSOCKRapidTask2018}. Refunction reuses containers between function invocations, reducing cold start times\cite{oliverstenbomRefunctionEliminatingServerless2019}. Pegurus also reuses warm containers between different function invocations to reduce cold start times\cite{liPagurusEliminatingCold2021}. Xanadu reduces cold start times by pre-warming containers based on a predictive model that estimates which subsequent functions will be triggered in a \faas{} workflow\cite{dawXanaduMitigatingCascading2020}.

\subsection{V8 isolate based isolation}
Whilst containers provide a much lighter footprint in comparison to VM based isolation, CloudFlare Workers\cite{CloudComputingContainers2018} and Vercel Edge Functions\cite{EdgeRuntime} both utilise V8 isolates to provide lightweight user-space isolation between functions. Whilst less secure than process-based isolation employed by contianers, V8 isolates provide a much denser packing of functions on a single host, and can provide much lower cold start times.

Additional work surrounding microarchitectural vulnerabilities such as Spectre within V8 isolates has been carried out to ensure that V8 isolates can execute securely in multitenant environment\cite{schwarzlRobustScalableProcess2022}.

Apache OpenWhisk\cite{apacheOpenWhisk2024} is an open-source \faas{} platform that provides container based isolation, leveraging Nginx as an HTTP gateway, Kafka as a message broker to queue invocations, CouchDB as a persistent data storage layer and OCI containers to execute arbitrary function logic.

OpenFaaS\cite{ellisOpenFaaS2024} provides container based isolation for \faas{} using Kubernetes to handle scaling and execution of functions, and Prometheus to handle scaling of the functions.

\subsection{GraalOS and GraalVM based isolation}
\todo[inline]{Discuss how GraalOS and GraalVM reduces cold start times.}

\subsection{Wasm based isolation}
WASM provides a secure and efficient environment for executing untrusted code\cite{WebAssembly} in a multitenant environment. Initially developed to run in the browser, WASM allows for the execution of untrusted code in a sandboxed environment without the need for a full VM or container. It has formal semantics\cite{haasBringingWebSpeed2017}, its embeddings can be formally proven to be memory-safe\cite{SecurefoundationsVWasm2024}, and utilises software based fault isolation to ensure that code cannot escape the sandbox\cite{SecurityWebAssembly}. Despite this effort, microarchitectural vulnerabilities such as Spectre still exist within WASM runtimes, and additional work is being carried out to mitigate these\cite{narayanSwivelHardeningWebAssembly2021}.

Fastly Edge Compute Platform\cite{EdgeCloudPlatform} utilises Lucet\cite{BytecodeallianceLucet2024} (now Wasmtime\cite{Wasmtime}) to sandbox WebAssembly executables from one-another. Fastly Edge Compute Platform cites startup times in the order of tens of microseconds to instantiate the sandbox.

In addition to JavaScript functions deployed using V8 isolates, Cloudflare Workers\cite{CloudComputingContainers2018} also allow WASM based functions to be deployed to their edge network. Unlike Fastly, Cloudflare Workers do not execute the function within a WASM runtime, and instead rely on the V8 runtime to execute WASM functions\cite{WebAssemblyWasmCloudflare2024}.

Sledge reduces cold-start times and increases throughput over other WASM based \faas{} frameworks by leveraging LLVM to compile WASM binaries, specifically targetting edge hardware and implements user-space scheduling of functions\cite{gadepalliSledgeServerlessfirstLightweight2020}.

FaaSM employs a similar approach to container based isolation, using CGroups and Namespaces for isolation, and executing WASM binaries. In order to improve interprocess communication and persisting state, FaaSM utilises shared memory to allow functions to communicate with one another\cite{shillakerFaasmLightweightIsolation2020}.

\section{Characteristics of \faas{} workloads}
A review of serverless use cases and their characteristics by Eismann et al.\cite{eismannReviewServerlessUse2020} found that the majority of serverless functions have shortlived executions on the order of milliseconds to seconds.

They found that the most popular languages to write serverless functions in were JavaScript, Python. Additionally, they note that the majority of applications consisted of 5 or less distinct cloud functions, indicating that large granularity is preferred for serverless functions.

Finally, they identified that the overwhelming majority of serverless functions interface with persistent block storage, and databases, accounting for 61\% and 47\% respectively.

\section{\faas{} billing models}
\label{sec:faas-billing-models}
\todo[inline]{Describe a deep dive on faas billing models}

\section{AWS Lambda}
AWS Lambda is by far the most popular\cite{eismannReviewServerlessUse2020,StateServerlessDatadog} \faas{} platform used by developers to deploy serverless applications.

\subsubsection{System architecture}

The AWS Lambda system architecture is centered around the concept of Firecracker MicroVMs\cite{agacheFirecrackerLightweightVirtualization2020}.

\begin{figure*}[t]
    \includegraphics[width=\linewidth]{node_modules/@faaas/aws-lambda-exec-env/assets/aws-lambda-exec-env.pdf}
    \caption{AWS Lambda Execution Environment}
    \label{fig:aws-lambda-exec-env}
\end{figure*}

\subsection{Concurrent executions}

Each invocation of a lambda function executes independently inside of it's own Firecracker MicroVM, in what is known as a slot.

Each MicroVM provides a slot which can handle a single invocation, however once this invocation completes, the slot can be used by another invocation.

\subsection{Pricing structure}

Billed for execution time from start to finish, since a MicroVM is provisioned the entire time.

\section{V8 and the Event Loop}
\label{sec:js-event-loop}

\todo[inline]{Outline the ins and outs of how event loops work. Also need to discuss re: function continuations.}

\section{Continuations and Continuation Passing Style}
A function continuation is a concept in programming\cite{sussmanSCHEMEInterpreterExtended1975}, reified as a datastructure encapsulating an execution state, and a function pointer that can be called to resume execution. They are used extensively across many languages, for example, they form the underpinnings of Rust's Futures API.
