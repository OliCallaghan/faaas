\chapter{Conclusion and Future Work}
\label{chap:conclusion}
\section{Conclusion}
During this research, we identified that many \faas{} workloads typically accept HTTP requests, construct a query and send it to a database for processing, before performing some form of preprocessing with the result, and returning it in the response.

Whilst awaiting the query result, functions are often idle yet are charged for the allocated but unused resources. From this observation, we propose executing functions in a split manner, whereby the first part of a function invokes an asynchronous request, and the second part processes the result. A cloud-provider-agnostic parameterised costing model is defined and then used to model various scenarios to determine the cost effectiveness of splitting functions.

Following this, we model response times of asynchronous requests using a Weibull distribution, and use this to determine the probability of a function split resulting in a cost reduction. We then use this probability to determine whether to split a function or retain the allocated resources and wait for the request to complete.

We then developed \faaasc{}, a system that automatically analyses and generates split \js{} \faas{} functions with minimal developer input, and evaluated the overheads associated with executing typical functions using different splitting strategies on AWS Lambda. Deploying to AWS, we developed a gateway, and a proxy around PostgreSQL that allows queries to invoke continuation functions passed to them once a query result is available.

We found that for OLAP workloads, split functions provide a 75\% cost reduction over executing the same function without splitting. We also found that for OLTP workloads, the overhead and penalty of splitting functions outweighs the cost reduction with the current AWS Lambda billing model.

We also introduce an adaptive splitting strategy, which deploys a monitor alongside a \faas{} function to monitor the response times of asynchronous requests and infers a response time distribution, which is used by executing functions to determine whether to defer to a proxy service to perform asynchronous requests, or execute the asynchronous request locally, retaining the allocated resources.

This adaptive splitting strategy was evaluated on AWS Lambda, and found to select the optimal splitting strategy, albeit with a short time-delay for the empirical response-time data to be aggregated and the response time distribution to be correctly inferred.

Finally, we consider a set of hypothetical scenarios with adjusted \faas{} billing models, and model how \faaas{} would perform under these scenarios.

\section{Future Work}
\label{sec:future-work}
In this section, we will discuss future work and directions that could be taken to extend the work presented in this thesis.

\subsection{Extending \faaasc{}}
A major unaddressed bottleneck of \faaas{} currently relates to the overhead of serialising and deserialising context state passed between function invocations. Whilst this is necessary in order to persist state between function invocation and handler, in the pathalogical case where a variable is declared at the start of the program, and only used at the end, the variable will need to be serialised and deserialised through each continuation.

This could be addressed in a variety of different ways, for eaxmple, a DAG of the JS program could be generated and the serialisation and deserialisation could be optimised to only include the variables that are used directly in the next continuation. Alternatively, all context could be written to a low latency cache such as Redis, and lazily loaded by continuations when they need it. This would mean that only a pointer to this data would need to be serialised and deserialised, rather than the data itself, however it would introduce additional network round-trips whenever data is accessed.

Additionaly, another bottlneck of the system is the latency whilst fetching the response time distribution parameters from the Redis cache in order to determine whether to locally invoke the asynchornous operation or delegate it to a proxy. This could be addressed by making a request directly to the proxy, which would internally determine an estimate for the asynchronous request. For instance, the database proxy could use the query estimate used by the plan optimiser to compute the probability that the function would be profitable if it split. In the case that the function should split, the proxy could then return the queue to serialise the context and the continuation to before exiting, otherwise if the split is not profitable, the proxy could just execute the query and directly return the result. This would reduce the latency by the round-trip time to the Redis cache.

\subsection{Potential solutions that would make OLTP workloads more viable}
In this section, we discuss potential solutions which would make OLTP workloads more viable on \faas{} platforms.

\subsubsection{Voluntary release of CPU time back to the hypervisor}
As discussed in Section \ref{sec:aws-lambda-execution}, \faas{} platforms such as AWS Lambda typically use virtual machines to isolate functions. The hypervisor is responsible for scheduling CPU time to each of the virtual machines and does so fairly, based on the resource allocation of the function to be executed.

From the results of the decribed experiments, it is clear that for the proportion of the time the function is executing, functions are allocated CPU resources which they are not using. The ability to voluntarily release this CPU time back to the hypervisor would allow the hypervisor to allocate this CPU time to other functions, and thus reduce the cost of executing the function, since only the memory would need to be allocated for the entire duration of the function's execution.

Freyr aims to accomplish a similar goal identifying overprovisioned CPU and memory allocated to serverless functions \cite{yuAcceleratingServerlessComputing2022} that are underutilised, and reclaiming this ahead of time. Whilst this showed a 40\% cost reduction in their experiments, it only performs this operation in a coarse-grained manner, since it does not adjust resource allocations over the course of a function invocation.

\subsubsection{Voluntary release of memory back to the hypervisor}
Whilst releasing CPU allocation back to the hypervisor when blocked on IO could reduce the cost of executing the function, memory would still need to be billed since it occupies memory that another function could be using. In the case that a function is blocked on IO, depending on the disk latency, it could be more cost effective to write the blocked function's memory to swap during the period that it is blocked, and then load the function back into memory once the response has arrived, and the node has capacity to continue execution of the function.

This would have the impact of increasing latency, as the function would need to perform a write-read cycle to and from disk, in addition to waiting until the hypervisor has capacity to load the function back into memory. This would however allow functions to be billed more accurately based on CPU time and memory that they have utilised.

%We propose a short experiment in order to assess the viability of this in AWS Lambda. By inspecting the CloudWatch logs from executing the experiment benchmarks, we observe that the average memory consumption for all the NodeJS functions was approximately \SI{108}{\mega\byte}. Since we cannot know the specifics of the disk used by the instances running the hypervisor to execute Lambda functions, we will assume that the write speed to the EFS (Elastic File System) from the Lambda environment is representative of the disk speed of the hypervisor. We deploy a simple function that writes and then reads a \SI{108}{\mega\byte} file of random data to and from EFS, and measure the time taken to perform this operation. This file is to be representative of uncompressed memory allocated to the MicroVM that is to be written to swap during a blocked IO operation. The results of this are shown in Figure \todo{run EFS IO benchmark}.

In order to achieve this technically, the operating system could notify the hypervisor that the function is blocked on IO when the operation system recieves an \verb|epoll\_wait| system call. The main difficulty with this approach would likely be the issue of correctly predicting when a call to \verb|epoll\_wait| would cause the function to yield for a long enough duration for it to be worth the hypervisor to swap the function out to disk.

Therefore, it may be more effective to implement support for this at a language level, whereby the language runtime would notify the hypervisor that the function is blocked on IO, and that the hypervisor should swap the function out to disk. If this occurs at the language level, then it would have access to context around the asynchronous IO operation, and could use this to predict its duration.

\subsubsection{Function continuation execution within the database}
We have so far focused on potential solutions that would allow for better resource utilisation of the hypervisor. Whilst an important component to the solution, it does not address the issues of latency encountered by performing the AWS Lambda slotting algorithm for each continuation. By formalising serverless functions as handlers which return continuations in Section \ref{sec:faas-cps}, we could pass the continuation directly to the database to execute after the query has been completed. This however still relies on low invocation costs in order to be cost effective.

Mainstream cloud providers typically use either container based isolation or VM based isolation, which both have relatively high invocation overhead. Other isolation mechanisms, such as the WASM isolation methods described in Section \ref{sec:faas-wasm-isolation}, have the potential to reduce invocation overhead, allowing continuations to be executed with a much lower cost penalty.

By embedding a WASM runtime into the database, it could even enable execution of the continuations returned by \faaasc{} handlers directly inside the database.

\section{Review of research into improving cold starts and invocation cost reduction for \faas{}}
\label{sec:faas-isolation}
In this section, we will review the current state of the art in research to reduce cold-start times and invocation costs in \faas{} platforms. We group reserach by the isolation mechanisms used by the \faas{} platform, and discuss the trade-offs between security and performance.

Typically, multi-tenany public \faas{} providers can be characterised into two categories: providers offering VM based isolation\cite{agacheFirecrackerLightweightVirtualization2020} and providers offering container based isolation\cite{GVisor}. There have been considerable efforts to reduce the invocation overhead associated with each of these classes.

\subsection{VM based isolation}
VM based isolation is typically considered a much more secure form of isolation between \faas{} functions on multitenant public clouds\cite{jithinVirtualMachineIsolation2014}, however it typically has the highest overhead, since each function runs in its own virtual machine, with its own kernel. This overhead usually manifests in higher cold start times, however there has been considerable work carried out to reduce this overhead\cite{razaviPrebakedUVMsScalable2015,agacheFirecrackerLightweightVirtualization2020,dawXanaduMitigatingCascading2020,oliverstenbomRefunctionEliminatingServerless2019}.

Kata Containers\cite{KataContainersOpen} provides a container-like interface to VMs, interfacing with existing hypervisors such as Firecracker\cite{agacheFirecrackerLightweightVirtualization2020}, QEMU\cite{QEMU} and Cloud Hypervisor\cite{CloudhypervisorCloudhypervisorVirtual} (previously NEMU\cite{IntelNemu2024}).

Firecracker \cite{agacheFirecrackerLightweightVirtualization2020} is a lightweight KVM\cite{KVM} based Virtual Machine Monitor (VMM) allowing MicroVM provisioning in the order of \qtyrange{125}{150}{\ms} that provides a secure and fast environment for running \faas{} functions.

MicroVMs are intended to be lightweight, and provide a secure environment for running functions, with a fast boot time. Firecracker is written in Rust and unlike QEMU, provides secure, fast and stripped back devices to the guest VM in order to both reduce the attack surface and improve performance\cite{jainStudyFirecrackerMicroVM2020}.

LightVM redesigns the Xen Hypervisor's control plane such that VMs can boot in the order of \qtyrange{2}{3}{\ms}\cite{mancoMyVMLighter2017}, and achieveing a much higher VM density on a host.

\subsection{Container based isolation}
In contrast to VM based isolation, container based isolation is typically considered to be less secure\cite{DemystifyingContainerVs}, however it has a much lower overhead, since each function runs in its own container sharing a kernel with the host\cite{WhatContainerDocker}. From a security perspective, any kernel vulnerabilities could potentially be exploited by a malicious container to escape the sandbox and gain access to the host\cite{linMeasurementStudyLinux2018}.

GVisor is a container runtime that aims to improve security for running containers in a multitenant environment\cite{GVisor}. It implements a userspace `application kernel' that intercepts system calls made by untrusted container, providing a layer of abstraction between any possible vulnerabilities in the host kernel and the code executing in the untrusted container.

Whilst cold-start times of container-based \faas{} isolation are considerably lower than VM based isolation methods, cold-starts times can vary based on a variety of factors. SOCK reduces cold-start times by caching common dependencies in container images, reducing the image footprint and improving cold start times\cite{oakesSOCKRapidTask2018}. Refunction reuses containers between function invocations, reducing cold start times\cite{oliverstenbomRefunctionEliminatingServerless2019}. Pegurus also reuses warm containers between different function invocations to reduce cold start times\cite{liPagurusEliminatingCold2021}. Xanadu reduces cold start times by pre-warming containers based on a predictive model that estimates which subsequent functions will be triggered in a \faas{} workflow\cite{dawXanaduMitigatingCascading2020}.

Apache OpenWhisk\cite{apacheOpenWhisk2024} is an open-source \faas{} platform that provides container based isolation, leveraging Nginx as an HTTP gateway, Kafka as a message broker to queue invocations, CouchDB as a persistent data storage layer and OCI containers to execute arbitrary function logic.

OpenFaaS\cite{ellisOpenFaaS2024} provides container based isolation for \faas{} using Kubernetes to handle scaling and execution of functions, and Prometheus to handle scaling of the functions.

\subsection{V8 isolate based isolation}
Whilst containers provide a much lighter footprint in comparison to VM based isolation, CloudFlare Workers\cite{CloudComputingContainers2018} and Vercel Edge Functions\cite{EdgeRuntime} both utilise V8 isolates to provide lightweight user-space isolation between functions. Whilst less secure than process-based isolation employed by contianers, V8 isolates provide a much denser packing of functions on a single host, and can provide much lower cold start times.

Additional work surrounding microarchitectural vulnerabilities such as Spectre within V8 isolates has been carried out to ensure that V8 isolates can execute securely in multitenant environment\cite{schwarzlRobustScalableProcess2022}.

\subsection{Wasm based isolation}
\label{sec:faas-wasm-isolation}
WASM provides a secure and efficient environment for executing untrusted code\cite{WebAssembly} in a multitenant environment. Initially developed to run in the browser, WASM allows for the execution of untrusted code in a sandboxed environment without the need for a full VM or container. It has formal semantics\cite{haasBringingWebSpeed2017}, its embeddings can be formally proven to be memory-safe\cite{SecurefoundationsVWasm2024}, and utilises software based fault isolation to ensure that code cannot escape the sandbox\cite{SecurityWebAssembly}. Despite this effort, microarchitectural vulnerabilities such as Spectre still exist within WASM runtimes, and additional work is being carried out to mitigate these\cite{narayanSwivelHardeningWebAssembly2021}.

Fastly Edge Compute Platform\cite{EdgeCloudPlatform} utilises Lucet\cite{BytecodeallianceLucet2024} (now Wasmtime\cite{Wasmtime}) to sandbox WebAssembly executables from one-another. Fastly Edge Compute Platform cites startup times in the order of tens of microseconds to instantiate the sandbox. The Wasmtime runtime now used by Fastly Edge Compute Platform employs a set of security features to ensure that code executing within the runtime cannot escape the sandbox, and that the runtime itself is secure\cite{SecurityWasmtime}. Wasmtime prepends a guard region before linear memories in order to protect against sign-extension bugs in the Cranelift JIT compiler resulting in invalid memory accesses between instances. It utilises guard pages on all thread stacks, triggering an exception if they are hit to prevent data leakage. Additionally, since Wasmtime is written in Rust, the surface which a vulnerability could manifest is reduced to only the unsafe sections of the codebase. The runtime is also implementing CFI mechanisms to use ARM specific instructions

In addition to JavaScript functions deployed using V8 isolates, Cloudflare Workers\cite{CloudComputingContainers2018} also allow WASM based functions to be deployed to their edge network. Unlike Fastly, Cloudflare Workers do not execute the function within a WASM runtime, and instead rely on the V8 runtime to execute WASM functions\cite{WebAssemblyWasmCloudflare2024}.

Sledge reduces cold-start times and increases throughput over other WASM based \faas{} frameworks by leveraging LLVM to compile WASM binaries, specifically targetting edge hardware and implements user-space scheduling of functions\cite{gadepalliSledgeServerlessfirstLightweight2020}.

FaaSM employs a similar approach to container based isolation, using CGroups and Namespaces for isolation, and executing WASM binaries. In order to improve interprocess communication and persisting state, FaaSM utilises shared memory to allow functions to communicate with one another\cite{shillakerFaasmLightweightIsolation2020}.
