\chapter{Ethical Considerations}
Whilst not immediately obvious from an ethical standpoint, there are a set of ethical considerations that do need to be made regarding the responsible usage of this research's output.

Most importantly, since this research delegates execution of asynchronous IO to proxies, it is important to consider the security implications of this. The proxies are responsible for executing the IO operations, and so have access to the data that is being transmitted. This means that the proxies must be trusted, and the data that is being transmitted must be considered to be secure. This is particularly important in the context of serverless functions, where the data that is being transmitted may be sensitive, and the proxies may be executing in a different security context to the serverless function.

Secondly, the research is focused on reducing the cost of executing serverless functions. This is important to consider in the context of the cloud providers that are being used. The cloud providers are providing a service, working under a set of assumptions on the types of \faas{} workloads, and so it is important to consider the impact that this research has on their business model and also the quality of service they can provide. Whilst it may have the benefit of reducing costs for the end user, it may also cause the cloud provider to increase the costs for everyone else, since functions have a higher average utilisation.

Finally, in terms of the environmental impact of this research, it is important to consider the impact that this research has on the energy consumption of the cloud providers. By reducing the cost of executing serverless functions, it may encourage the execution of more serverless functions, which may increase the energy consumption of the cloud providers. This is particularly important in the context of the environmental impact of cloud computing, and the need to reduce the energy consumption of data centres.

Overall, this research aims to reduce the cost of running \faas{} workloads, which inevitably will target smaller business and individuals who are looking to reduce their costs, and primiarly rely on \faas{} platforms to provide flexible low-cost pricing, so that they can compete with larger businesses.
